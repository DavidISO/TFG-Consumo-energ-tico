\section*{Descripción de el entrono}

Todo el proyecto va orientado a la empresa Coabre, con origen en Silla (Valencia).
 Se trata de una empresa a la que el ssitema de automatización ha sido realizado por la empresa donde realizo mi FORTE.
 Por lo que es muy conveniente que sea esta empresa la que realice la nueva funcionalidad. 

Topología de la empresa.

La empresa se inspira en un modelo cliente - servidor.

●	Dentro de el servidor se encuentran dos subsistemas de gran calado e la empresa SUITER y REPORTER.
●	De la parte del cliente tenemos browser, cliente de fábrica y HMI (Human Machine  interface). Dentro de HMI está:
	○	Piquera
	○	Truck
	○	Premezcla
	○	Dos. Manual
	○	Trasvases 

\section*{Sistema de automatización SUITER}

La empresa dispone de un servicio de automatización de procesos llamado SUITER, este 
sistema. Este sistema se encarga de automatizar el proceso industrial, como podría ser
 gestión de molinos, gestión de la entrada de materias en los silos. Todo esto requiere 
como es de prever una gestión en tiempo real.

Queda destacar que se tratan de bases de datos MySQL.

Al suiter le da soporte una base de datos nombrada con el nombre de la fábrica(Coabre).
 Esta base de datos contiene registros dónde SUITER accede de forma concurrente.


\section*{Sistema de reporte REPORTER}

El REPORTER es el sistema encargado de presentar una página web, desde dónde
 poder gestionar el proceso productivo, este sistema dispone de dos bases de datos,
 Coabre reporter, que da soporte al servicio web, además de una base de datos replicada
 de la de Coabre a la que accede para realizar consultas pero esta vez no de tiempo real, 
ya que una página web no es un sistema de riesgo, además de escribir los datos de
 entrada cómo puede ser la entrada de materias primas, las mezclas... El motivo de
 una base replicada viene por el segundo objetivo del REPORTER que es obtener informes. 
Estos informes suelen tener una grán carga de datos, por lo que ralentizará al SUITER
 en el que una desviación en segundos puede suponer grandes desviaciones en el proceso.
 imaginémonos que queremos vaciar un silo si tardamos un segundo más en cerrar la compuerta
 la fórmula variaría en kilos, algo que no es admisible. Al final tendremos tres bases de datos:
 coabre_server como slave, coabre como master y reporter_coabre para la gestíon de la pagina web.

Al ser el Reporter el que administra la página web también entra en la parte de Browser,
 el Browser se trata de servidor web Apache Tomcat. En esta parte la comunicación se 
realiza mediante peticiones ajax.

En el Browser corre en un tomcat como un contenedor de servlets. Toda la interfaz de 
interaccion con el cliente esta realizada con java script, jsp, html, jtls y css para darle estilo.
 El servidor estaría realizado básicamente en java.

\section*{Protocolo de comunicación}

los estados por los que pasa una consulta del cliente son los siguientes. El cliente tiene 
conocimiento de todos los servlets almadenados por el contenedor tomcat que se tratan 
de las distintas opciones que puede realizar. A partir de ahi se manda una petición
 request http que resuelve el web.xml de tomcat devolviendo la ruta del servlet solicitado.
 Pero antes de devolver la ruta y proceder a resolver la petición tiene que pasar un filtro
 basado en dos partes: inFIlter y genericFilter. El inFilter recibe el ido de ususario y comprueba
 que tiene acceso a esa opción de forma que filtra los permisos. El genericFilter establece 
los parámetros de sesión. Una vez pasa el filtro entramos a resolver la petición del cliente.

El servidor trabaja basándose en el modelo en tres capas de gestión de base de datos: 
persistencia, negocio y presentación. Presentación que corresponde a la interacción directa
 con el clientes son los servlets, tras ellos está el lib que corresponde a la capa de lógica
 de negocio y el dao (minúscula) y Dao (mayúscula) que se encargan de la interaccion de
 la base de datos, de forma más directa el Dao que genera las consultas SQL.

Siguiendo el diagrama de estados en el que nos habíamos quedado, al acceder al servlet se
 genera un lib que contiene todas las funciones que puede realizar el cliente. Estas funciones 
son realizadas por el dao que pide al Dao que realice la conexión con la base de datos. 
El servlet se encarga de añadir este lib a unas variables de contexto a las que accede el cliente.
 A parte existen peticiones ajax que genera el cliente.

El lib es el caota los datos de el formulario, el servlets carga los privilegios y las opciones, 
y el dao es el que contine la lógica de negocio. Jquery

\section*{Objetivo general}

Como inicio convendría saber a grandes rasgos en que entorno nos estamos moviendo.
 Este proyecto se enmarca en el ámbito de la fabricación de piensos de una empresa.
 Se trata de un entorno industrial donde necesitamos asegurar una trazabilidad de todos los
 procesos que se realizan. Como primera aproximación, el proceso de fabricación de piensos
 sería, la entrada de materias primas y la salida de piensos o materias. Para poder hacer
 un seguimiento de el proceso productivo, lo dividimos en lotes. Los lotes nos permiten
 seguir la trazabilidad de las materias primas.

Pongamos un ejemplo: A la empresa le entra 1000Kg de cebada, entonces esa cebada la
 anotaremos como LE1 (Lote de entrada 1) este lote nos permite hacer el seguimiento 
desde que sale de el camión hasta que se almacena para posteriormente ser usado.
 Parte de este lote podría ser usado para mezclarlo con otros lotes y formar un pienso,
 esto se llama dosificación. Una vez el lote sale de su almacenamiento pasaría a ser 
LF1 (Lote de fabricación 1) que puede contener otros lotes de entrada. 
El proceso seguiría de la misma forma con todos los procesos de los que disponga la fábrica.

A partir de aquí, conociendo lo que es un lote, pasamos a describir el objetivo de el proyecto.
 El objetivo es obtener el consumo energético por cada lote. Esto se dividirá en dos partes:
 Elaboración de un cliente que procese los datos de los que ya se dispone para presentárselos
 a el cliente en una página web y el desarrollo de una herramienta de generación de informes.

\section*{Clientel}

Tendrá dos tareas principales, mostrar el consumo energético por lote donde se puede ver u
n cronograma de los motores que intervienen en un lote y comparar los resultados obtenidos
 con el consumo real de un contador fiscal con el fin de ajustar el error cometido obteniendo 
consumo auxiliar y consumo ponderado.

Al poder acceder a la base de datos de el servicio de automatización de la empresa, podremos
 obtener información relevante a los lotes en tiempo real. También dispone de el consumo 
energético de cada motor que intervenga en el proceso productivo.

Para realizar esta tarea necesitamos el soporte de una base de datos,
 que en nuestro caso es MySql, crearemos tres nuevos registros:
•	reg_logs
•	reg_consumo_cuartohorario
•	reg_orden
Haremos uso de las tablas:
•	reg_contaodres
•	str_elementos
•	tr_lotes
•	tr_rel_lotes

El cliente al realizar la conexión con el servidor obtiene los objetos nodo que intervienen en la 
conexión, estos objetos contienen los datos de los motores y de los lotes. La comunicación se 
realiza por eventos.

\section*{Captura de logsl}

En logs se registrará los cambios de estado de los motores asociados a un lote, junto con 
una marca de tiempo para registrar en qué momento se produjo el cambio de estado. Los motores
 pueden estar en tres estados: arranque, paro y alarma. A efectos de el cálculo de el consumo
 energético, una parada y una alarma significan los mismo: el motor deja de funcionar por lo que 
dejaría de consumir. Pero sería interesante mostrar un cronograma de los logs incluyendo los 
momentos en los que se han producido alarmas. El cliente captura eventos generados por el
servidor SUITER, cada vez que se genera un cambio de estado en los motores, se encolarán
 usando el modelo de concurrencia productor – consumidor. El consumidor se trata de un proceso
 en nuestro programa cliente. El factor de degradación es un valor que inicialmente se pondrá
 a 1, pero que no va a ser usado en este proyecto, tan solo con vistas a futuro. La potencia 
de degradación y la potencia en vacío será utilizada para el cálculo de el consumo energético. 
Inicio es la fecha en la que se instaló el motor.

El modelo entidad relación sería:

-	reg_logs(codigo, id_elemento, id_lote, arrancado, alarma, potencia_nominal, potencia_vacio, factor_degrdacion, incio, ts)
-	reg_logs.arrancado. reg_logs.alarma reg_logs.ts NOT NULL
-	reg_logs.arrancado {0,1}
-	reg_logs.alarma {0,1}

\section*{Obtención de un consumo cuartohorario}

reg_consumo_cuartohorario

Dicha tabla es necesaria para poder obtener el consumo de cada lote cada 15min, esta 
decisión se debe a que los contadores registran la energía real consumida cada cuarto horario.

Esta tabla nos permitirá comparar el consumo obtenido con el consumo de los contadores, 
con el fin de ajustar el coste obtenido ponderando de todos los lotes en un cuarto de hora.

Antes de realizar la comparación con el contador tendremos que rellenar la tabla con
 los kw de carga y los kw en vacío, de esto se encargará un proceso del Cliente. Pero
 surge un problema con el consumo en vacío producido ya que las empresas suelen 
dejar en cedidos los motores antes que pararlos y arrancarlos debido al gran consumo
 de él arranque, aunque siempre hay un tiempo límite en el que el motor puede estar
 en vacío. ¿cómo detectamos que está en vacío?, pues cuando id_lote esté a null y 
arrancado en 1 (true)

¿A quién debemos imputarle el consumo en vacío a el lote que sale o a el que entra?
 Pongamos un ejemplo donde se pueda ver más claramente:

Imaginemos nos que el molino está moliendo un LF1(Lote de fabricación) pero sale de
 el molino, el motor del molino sigue arrancado y hasta dentro de 10 min no llega el otro
 lote LF2. En estos casos hemos tomado la decisión de imputar el consumo en esos 10min
 a el lote entrante LF2 ya que si no estuviera en cola el motor acabaría parando.
 También se podría producir las sucesión de estados: estado1   (id_lote= null, arrancado=1) ,
 estado2  (id_lote= null, arrancado=0), esto indica que el lote ha salido, no va a entrar ningún 
otro lote y el motor supera su tiempo de arrancado en vacío; ese tiempo se le imputaría
 a el lote saliente LF1.

Además tiene una última columna que dispone de los kw reales, que contiene el consumo
 real obtenido gracias a la comparación con los contadores, de lo que hablaré más adelante.

Modelo entidad relación de reg_consumo_cuartohorario

reg_consumo_cuartohorario(codigo, id_lote, kw_carga, kw_vacio, kw_reales, ts)
reg_consumo_cuartohorario.id_lote NOT NULL

reg_contadores

En reg_contadores obtenemos el consumo por cada contador en una fecha, cada cuarto de hora
, pero todos los contadores no nos interesan, para discriminar entre contadores necesitamos la
 tabla str_elemento que contiene el tipo de contador.

De la tabla reg_contadores necesitaremos reg_contadores.id_contador, reg_contadores.fecha,
 reg_contadores.eae (consumo) y de str_elemento será necesario str_elemento.codigo y 
str_elemento.nombre.

Modelo:

reg_contadores(id_contador, fecha, eae)
str_elemento(codigo, nombre)

reg_contadores.id_contador  str_elemento

\section*{Obtención de el consumo por cada orden}

Una vez obtenido el consumo por cada lote en un mismo cuarto horario y el consumo real en un cuarto 
horario, podemos ponderar. Sumando el consumo de cada lote tenemos el consumo obtenido en un 
cuarto de hora x que lo comparamos con el consumo real de los contadores que intervienen,
 obtendremos una desviación que la tendremos que dividir entre los lotes que intervienen en ese 
cuarto horario ponderando. Todo este cálculo lo realizará un solo proceso.

El resultado de esta operación irá a reg_consumo_cuartohorario.kw_reales


Finalmente queremos obtener el consumo total de cada lote junto con el heredado de otro lote, este
 consumo heredado se debe a las dosificaciones. Pongamos un ejemplo:

Supongamos que tenemos un lote de maíz LE1 que dosifica 5 de sus 10 kg a un LF1 entonces el lote LF1 
tendrá el consumo en procesar esos 5Kg más el consumo heredado de LE1 que es el consumo de la entrada
 de ese maíz dividido entre dos.

Por lo tanto, del modelo entidad relación de reg_orden es:

reg_orden(codigo, id_lote, kw_proceso, kw_heredado,ts)
reg_oreden.id_lote, reg_oreden.ts NOT NULL
reg_oreden.id_lote UNIQUE

Pero surge una posible excepción, el lote que ha dosificado puede cambiar, lo que hace que el consumo 
heredado cambie, además ese consumo puede cambiar en los lotes que dosificaron a el lote que ha 
dosificado, es decir, hacia arriba en la gerarquía. También cambia en los hijos de este Lote.

Generalmente se genera una jerarquía en modo de árbol, dónde la raíz empieza la dosificación. Entonces 
los pasos a seguir serían los siguientes, cada cierto tiempo se revisa el ts de el lote que ha dosificado,
 si cambia habrá que cambiar el el consumo por kilo de todos los nodos del árbol que le dosificaron hacia
 arriba hasta llegar a la raíz, en caso de que el que cambia es la raíz no habrá que recorrer el árbol hacia
 arriba. Además, hay que cambiar el consumo hacia abajo hasta llegar a las hojas, en el caso de que sea 
una hoja, solo se cambia ella y finalizaría.

Una vez tengamos todos estos datos, lo más interesante sería ver los lotes más próximos a la venta ya 
que contienen el consumo final y no van a tener que heredar.





